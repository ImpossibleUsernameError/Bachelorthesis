%%%%%%%%%%%%%%%%%%%%%%%%%%%%%%%%%%%%%%%%%%%%%%%%%%%%%%%%%%%%%%%%%%%%%%%%
\chapter{Implementierung der Relay Attacke}
\label{sec:problemdescription}
%%%%%%%%%%%%%%%%%%%%%%%%%%%%%%%%%%%%%%%%%%%%%%%%%%%%%%%%%%%%%%%%%%%%%%%%

Ziel dieses Kapitels ist die Beschreibung der praktischen Durchführung der NFC Relay Attacke. Es soll gezeigt werden, dass es mithilfe weniger Modifikationen am Android Betriebssystem möglich ist, die NFC Signale eines POS Terminals an ein zweites mobiles Gerät weiterzuleiten und auf diesem eine mobile Zahlung zu initiieren. Es wird dargestellt, auf welche Art und Weise ein Eingreifen in die Kommunikation der Zahlungsanwendung mit der NFC Schnittstelle möglich ist und wie die Antwortsignale der Zahlung an das weit entfernte POS Terminal zurückgesendet werden können. 

Ebenfalls soll diskutiert werden, welche unterschiedlichen Komponenten zum Aufbau und zur Durchführung der Relay Attacke notwendig sind, wie diese implementiert wurden und welche Rolle sie bei der Durchführung der Attacke einnehmen. Auftretende Schwierigkeiten sowie getroffene Entscheidungen sollen ebenfalls dargelegt werden.

Am Ende dieses Kapitels soll eine funktionsfähige Proof-of-Concept Implementierung der Relay Attacke stehen, deren Umsetzung mit einfachen Methoden des Android Frameworks realisiert wird und die als Grundlage für weiterführende Forschungen dienen soll. 

\section{Aufbau der Implementierung}

Zur Durchführung der Attacke wurden zwei handelsübliche, NFC-fähige Android Smartphones für die Rolle des Angreifers sowie des Opfers verwendet. Auf dem Angreifer-Gerät, einem HTC U11, war hierbei die Android Version 8.0.0 (API 26) vorhanden, auf dem Opfer-Gerät, einem Google Nexus 5X wurde eine modifizierte Version des Android Open Source Projektes mit der Version \textbf{XXX} installiert. Welche Modifikationen an diesem Betriebssystem vorgenommen wurden, wird in Kapitel \textbf{XXX} detailliert beschrieben.

Umgesetzt wurde die Relay Attacke durch die Entwicklung einer Proof-of-Concept Hostcard Emulation App, die eine handelsübliche mobile Payment Applikation wie beispielsweise Android Pay simulieren soll. Diese Anwendung wurde auf dem Opfer-Gerät installiert und als Ziel der Attacke sollte die HCE App zur Durchführung einer Zahlung und Weiterleitung der Daten an den Angreifer angeregt werden, ohne dass sich das Opfer in der Nähe eines POS Terminals befindet. Die Entwicklung der HCE App wird in Kapitel \textbf{XXX} dargelegt. 

Die Kommunikation der Geräte wurde durch eine weitere mobile Anwendung, welche unabhängig von der HCE Applikation operiert, realisiert. Diese existiert sowohl auf dem Angreifer- als auch auf dem Opfer-Gerät und sorgt für eine Verbindung zwischen den Kommunikationspartnern über ein Wireless Local Area Network (WLAN). 

Diese Anwendung kommuniziert darüber hinaus auch mit dem Android Betriebssystem und sorgt für die Ausführung der HCE App, sobald ein Signal vom Angreifer-Gerät empfangen wird. Darüber hinaus wird auch die Antwort der HCE App, die eigentlich an die NFC Schnittstelle gesendet wird, ebenfalls von der Kommunikationsanwendung abgefangen und an den Angreifer weitergeleitet. Mit welchen Mechanismen die unterschiedlichen Kommunikationswege umgesetzt wurden, wird in Kapitel \textbf{XXX} genauer ausgeführt. 

Zur erfolgreichen Durchführung der Attacke sind Modifikationen am Android Betriebssystem notwendig. Diese sorgen dafür, dass die entwickelten Anwendungen die passenden Signale erhalten, um korrekt operieren zu können. Bei der Durchführung der Attacke wird angenommen, dass eine modifizierte Version des Android Source Codes auf dem Opfergerät installiert werden kann.
Die genaue Durchführung der Änderungen am Android Betriebssystem wird in Kapitel \textbf{XXX} beschrieben. 

In folgender Abbildung ist eine Übersicht der Relay Attacke und der zur Durchführung entwickelten Komponenten gegeben. 

\begin{figure}
	\centering
	\caption{Übersicht der Relay Attacke}
\end{figure}

In Abbildung \textbf{XXX} ist in erster Linie ein POS Terminal sowie ein Angreifer- und ein Opfer-Gerät dargestellt. Das POS Terminal wird durch eine Terminal Simulator Software auf dem Computer dargestellt. Über eine Universal Serial Bus (USB) Verbindung wird von einem NFC Reader die NFC Funktionalität zur korrekten Funktionsweise der Terminal Software bereitgestellt. 

 Bei Kontakt mit dem Angreifer wird vom POS Terminal eine kontaktlose Zahlung über NFC eingeleitet. Die Commmand Apdus werden hierbei direkt in der Kommunikationsanwendung auf dem Angreifer-Gerät empfangen. Diese nutzt nach dem Empfang eines Kommandos die WLAN Verbindung, um das Apdu an das Opfer weiterzuleiten. Die Kommunikationsanwendung, die auf dem Opfer-Gerät standardmäßig im Hintergrund aktiv ist, empfängt die Signale, woraufhin sie eine Kommunikation über das Betriebssystem mit der auf dem Opfer vorhandenen HCE Applikation startet. Die Rolle dieser HCE Applikation kann im Prinzip von jeder handelsüblichen Mobile Payment Anwendung übernommen werden. Diese Anwendung ist das eigentliche Ziel des Angriffes. Nachdem sie vom Betriebssystem gestartet wurde, wird eine mobile Zahlung in die Wege geleitet. 
Bei der Durchführung der Zahlung von der Zahlungsapplikation Response Apdus an die NFC Schnittstelle des Gerätes gesendet, nachdem die HCE Anwendung davon ausgeht, durch NFC aufgerufen worden zu sein. 
Das modifizierte Betriebssystem leitet nun diese Antworten an die Kommunikationsanwendung zurück, welche diese über WLAN wiederum dem Angreifer mitteilt. Vom Angreifer-Gerät werden die Apdus schlussendlich an das POS Terminal zurückgeleitet und die Relay Attacke wurde erfolgreich durchgeführt. 

In den nachfolgenden Kapiteln wird ausgeführt, wie die einzelnen Komponenten im Detail umgesetzt wurden, um die beschriebene Funktionalität bereitzustellen. 

\subsection{Die Hostcard Emulation Anwendung}

Das Android Framework stellt zur Implementierung einer Hostcard Emulation Anwendung die Klasse HostApduService zur Verfügung. Diese Android-Service Klasse sorgt für das Annehmen von Command Apdus, nachdem diese von der NFC Schnittstelle empfangen wurden. Um diese empfangenen Apdus verarbeiten zu können, muss die Klasse erweitert und die processCommandApdu Methode überschrieben werden. 
Welche Applikation zur Verarbeitung eines eintreffenden Command Apdus gestartet werden soll, wird vom Betriebssystem festgestellt. Dieses besitzt einen Routing Mechanismus, der auf den sogenannten Application IDs (AIDs) aufgebaut ist \cite{androidHce}. Das sind Identifier, die Zahlungsanwendungen eindeutig identifizieren und sie müssen bei der Installation einer Mobile Payment App von dieser beim Betriebssystem durch eine XML-Datei in der Anwendung registriert werden \cite{androidHce}.

Die in dieser Arbeit dargestellte Proof-of-Concept Umsetzung einer HCE Applikation verwendet die AID einer Mastercard MAESTRO Debitkarte, die durch die Hexadezimal Darstellung A0000000043060 identifiziert wird. Gemeinsam mit den Standard Identifiern für Zahlungssysteme, die noch vor der eigentlichen Anwendung ausgewählt werden, wird die MAESTRO-AID in der aids.xml Datei festgelegt. 

\begin{figure}
	\caption{Festlegen der AIDs in der aids.xml Datei}
\end{figure}

Diese Datei wird bei der Registrierung der HostApduService Klasse in der AndroidManifest.xml Datei referenziert, um dem Betriebssystem mitzuteilen, welche AIDs von der HCE Anwendung verarbeitet werden können. 

Wird eine Smartcard bzw. ein NFC fähiges Mobilgerät, auf dem eine Zahlungsanwendung vorhanden ist, in die Nähe eines POS Terminals gebracht, so wird von diesem das aus den Grundlagen bekannte SELECT PPSE Kommando gesendet. Nach der Antwort, welche AIDs verarbeitet werden können und dem SELECT AID Kommando wird die implementierte HCE Anwendung über die erweiterte HostApduService Klasse aufgerufen. 
Nach Auswahl der Applikation zur Verarbeitung der Zahlung werden, solange die NFC Verbindung nicht unterbrochen wird, alle nachfolgenden Apdus ebenfalls an diese Klasse weitergeleitet \cite{androidHce}. 

Um eine echte Zahlungsanwendung zu simulieren wurden in der implementierten App die ersten Response Apdus einer echten MAESTRO Debitkarte gespeichert. Als Reaktion auf ein ankommendes Command Apdu wird deshalb das passende Response Apdu an das Betriebssystem zurückgeliefert. 

\subsection{Die Kommunikation zwischen Angreifer und Opfer}

Um die vom Angreifer-Gerät erhaltenen Apdus an das Opfer weiterleiten zu können, muss zwischen den beiden Geräten eine sekundäre Verbindung hergestellt werden. Moderne Smartphones besitzen für den Aufbau einer derartigen Kommunikation mehrere unterschiedliche Möglichkeiten:

\begin{itemize}
	\item \textbf{Bluetooth}: Eine drahtlose Kommunikationsmöglichgeit zur Übertragung von Daten zwischen zwei Geräten. Seit Ende des Jahres 2016 ist der BLuetooth Standard in der Version 5 verfügbar \cite{bluetoothPressRelease}. Seit dieser Version ist mit Bluetooth eine Übertragungsdistanz von 40 Metern mit einer maximalen Geschwindigkeit von 2 Mbit/Sekunde möglich \cite{bluetooth5}. 
	Um eine Verbindung über Bluetooth mit zwei Android Geräte aufzubauen, müssen diese bei ihrem ersten Verbindungsaufbau gekoppelt werden, was durch den Benutzer des Gerätes über ein Dialogfeld bestätigt werden muss \cite{android-bluetooth}. 
	\item \textbf{Mobiles Internet}: Jedes Smartphone besitzt die Möglichkeit, eine mobile Internetverbindung herzustellen. Die Datenübertragungsgeschwindigkeit kann bei modernen Geräten ebenfalls mehrere hundert Mbit/Sekunde betragen \cite{lte}. 
	\item \textbf{WLAN}: Über ein Wireless Local Area Network kann eine lokale Drahtlosverbindung zwischen Geräten in näherer Umgebung hergestellt werden. Eine WLAN Verbindung wird typischerweise von einem Zugriffspunkt, wie einem WLAN Router realisiert, der die Verbindung zwischen den Geräten sowie meistens eine Verbindung zum Internet bereitstellt. Auf modernen Smartphones kann ein derartiger Zugriffspunkt auch über einen Hotspot erstellt werden. Die Übertragungsgeschwindigkeit sowie die -distanz hängen bei WLAN sehr stark von den verwendeten Geräten sowie Zugriffspunkten ab. Mehrere hunderte Mbit/Sekunde über Distanzen von 20 bis 500 Metern sind allerdings durchaus typische Größen. 
	Befinden sich zwei Android Geräte im selben lokalen WLAN-Netzwerk ist für einen Verbindungsaufbau keine Benutzerinteraktion erforderlich. 
	\item \textbf{Wi-Fi Direct (WifiP2P)}: Mobile Geräte, die Wi-Fi Direct unterstützen, sind in der Lage, eine Wireless Fidelity (Wi-Fi) Verbindung untereinander aufzubauen ohne einen zentralen Zugangspunkt wie einen Hotspot oder Router zu benötigen. Mit Wi-Fi Direct sind Übertragungsdistanzen von 200 Metern mit Geschwindigkeiten von bis zu 250 Mbit/Sekunde möglich \cite{wifidirect}. Beim ersten Aufbau einer Verbindung von zwei Android Geräten über Wi-Fi Direct muss diese wie bei Bluetooth vom Benutzer des Gerätes bestätigt werden. 
\end{itemize}

Um eine Kommunikationsverbindung zwischen Angreifer und Opfer bereitzustellen, wurde in der implementierten Kommunikations App zunächst versucht, eine Verbindung zwischen den Geräten über das mobile Internet herzustellen. Um dies zu ermöglichen wurde auf dem Opfer-Gerät eine Server-Anwendung implementiert, die beim Starten der Kommunikationsanwendung einen Kommunikationsendpunkt (Socket) einrichtet. Über das TCP/IP Protokoll wird durch diesen auf eingehende Verbindungen gewartet. Konnte eine Verbindungsanfrage erfolgreich angenommen werden, wird zwischen den beiden Geräten eine TCP/IP Verbindung eingerichtet. Über diese können Nachrichten ausgetauscht werden. 
Auf dem Angreifer-Geräte wurde in Analogie zur Serverkomponente ein TCP/IP Client-Kommunikationsendpunkt erstellt, der versucht, eine Verbindung zum Server herzustellen. 
Dieser Versuch des Verbindungsaufbaus verlief allerdings erfolglos aufgrund der Tatsache, dass Smartphones eine private IP-Adresse im mobilen Internet zugewiesen bekommen, die von außen nicht erreichbar ist. Nachdem auch der Router, der die Internetverbindung für das Smartphone bereitstellt, nicht zugänglich ist, kann diese Möglichkeit nicht verwendet werden, um eine direkte drahtlose Verbindung zwischen zwei mobilen Geräten herzustellen \cite{androidtcpinternet}. 

Würde allerdings die Serveranwendung auf einem über das Internet zugänglichen Gerät gestartet werden und sowohl Angreifer als auch Opfer mit diesem Gerät verbunden werden, könnte das mobile Internet über diese dritte Instanz zum Austausch von Informationen genutzt werden.

Nachdem der Versuch, eine Verbindung über TCP/IP und das mobile Internet fehlschlug, wurde als nächste Verbindungsmöglichkeit Wi-Fi Direct herangezogen. Um eine derartige Verbindung aufzubauen, war es notwendig, einen BroadcastReceiver zu implementieren, der auf bestimmte Wi-Fi Direct Aktionen reagiert. Auf dem Angreifer-Gerät wurde danach eine sogenannte Peer Discovery durchgeführt, die als Ziel hatte, alle in der Nähe befindlichen Wi-Fi Direct Geräte zu finden und aufzulisten. Wurde die Liste von Geräten erfolgreich geladen, konnte eine Verbindung zum Opfer aufgebaut werden. Hierfür mussten die Verbindungsinformationen des Opfers bekannt sein, da ansonsten keine Möglichkeit bestand, die Geräte anhand der Verbindungsinformationen zu identifizieren. Wurde die Verbindung erfolgreich hergestellt, wurde die Applikation über den BroadcastReceiver und eine dafür vorgesehen Aktion benachrichtigt. Daraufhin konnten die Komponenten, die für die Verbindung über TCP/IP vorgesehen waren, mit den Verbindungsinformationen des Opfer-Gerätes wiederverwendet werden. Zu bemerken ist, dass beim ersten Erstellen der Verbindung, diese in einem Dialog auf dem Opfer-Gerät akzeptiert werden musste.
Nach erfolgreicher Durchführung dieser Schritte war es möglich, Nachrichten zwischen dem Angreifer und dem Opfer zu versenden. 

Diese Verbindung war allerdings relativ unzuverlässig. Ein Grund dafür war, dass das Wi-Fi Direct mehrmals in einen inaktiven Zustand wechselte. Ein Versuch, das Wi-Fi Direct aktiv zu halten, indem eine ständige Peer-Discovery in einem anderen Thread ausgeführt wurde, erzielte ebenfalls nicht die gewünschte Verbindungsqualität. 

Nachdem auch nach mehreren Versuchen, keine durchgehende Wi-Fi Direct Verbindung über längeren Zeitraum hergestellt werden konnte, wurde schlussendlich die WLAN Technologie als Kommunikationskanal für die Relay Attacke ausgewählt. Diese verspricht eine zuverlässige, schnelle Verbindung auch über größere Distanzen im selben lokalen Netzwerk. Darüber hinaus, muss beim Verbindungsaufbau keine Benutzer-Interaktion erfolgen. 

Für eine Kommunikation zwischen den Geräten über ein WLAN-Netzwerk konnten die Komponenten, wie sie für eine Implementierung über das mobile Internet anfangs entwickelt wurden, ohne Änderungen übernommen werden, nachdem nur die IP-Adresse des Opfers geändert werden musste. 

\subsection{Interception der Android NFC Schnittstelle}

Um zu verstehen, wie die Kommunikation mit der Android NFC Schnittstelle auf das Angreifer-Gerät umgelenkt wird, muss zunächst ein grundlegendes Verständnis der Host Card Emulation Funktionalität auf Android Smartphones geschaffen werden. Die Architektur ist in nachfolgender Abbildung veranschaulicht und wird im darunterstehenden Unterkapitel genauer beschrieben. 

\begin{figure}
	\caption{Android NFC Host Card Emulation Architektur}
\end{figure}

\subsection{Die Android NFC Host Card Emulation Architektur}

Im Android Betriebssystem finden sich 2 Pakete, die für die Funktionalität von Host Card Emulation von besonderer Bedeutung sind: 

\begin{itemize}
	\item \textbf{frameworks/base/core/java/android/nfc}: Beinhaltet die Komponenten, die vom Android Framework für die mobile App-Entwicklung bereitgestellt werden. Die Klassen in diesem Paket können von jedem Entwickler beim Entwickeln einer App zur Nutzung der NFC Funktionalität verwendet werden. Dieses Paket wird in der nachfolgenden Beschreibung als "Framework" bezeichnet.
	\item \textbf{packages/apps/Nfc}: Dieses Paket enthält die NFC Funktionalität auf Systemservice Ebene. Grundsätzlich stellen die Java Klassen eine Android Systemapplikation zur Verfügung, die die NFC Funktionalität bereitstellt. Die Klassen in diesem Paket werden von den Android Framework Komponenten über den Binder IPC Mechanismus angesprochen. Dieses Paket wird in der nachfolgenden Beschreibung als "Systemapplikation" bezeichnet.
\end{itemize}

Verfolgt man einen Top-Down Ansatz bei der Analyse der Host Card Emulation Funktionalität, befindet sich der Einstiegspunkt in der HostApduService Klasse, die sich im Frameworks-Paket befindet und somit in jeder beliebigen Android Applikation verwendet werden kann. Um HCE Funktionalität zur Verfügung stellen zu können, muss dieses Android Service von der App erweitert und im AndroidManifest.xml gemeinsam mit den unterstützten Application IDs registriert werden. Danach können Apdus über NFC von der Applikation in der processCommandApdu Methode empfangen werden. 

Die Frage, die sich zunächst stellt ist, wie die vom NFC Controller empfangenen Apdus an die HostApduService Klasse weitergeleitet werden. Bei der Untersuchung der HostApduService Klasse stellt sich heraus, dass die processCommandApdu Methode innerhalb eines MessageHandlers aufgerufen wird. Dies ist eine Klasse des Android Binder IPC Mechanismus. Daraus lässt sich schließen, dass ebenfalls eine Instanz eines Messengers, der für die Kommunikation mit dem HostApduService verantwortlich ist, im Systemapplikation-Paket zu finden ist. 

Diese Messenger Instanz befindet sich in der HostEmulationManager Klasse im Systemapplikation-Paket. Diese Klasse ist verantwortlich für die korrekte Weiterleitung sowohl von Command- als auch von Response-Apdus. Um diese Funktionalität bereitzustellen, besitzt die Klasse das bereits angesprochene Messenger-Objekt, um über dieses Command-Apdus an eine HostApduService-Instanz weiterzuleiten. Response-Apdus werden analog auf umgekehrtem Weg empfangen. Hierzu besitzt der HostEmulationManager ein MessageHandler-Objekt, das über die dazugehörige Messenger-Instanz im HostApduService angesprochen wird. Darüber hinaus verfügt der HostEmulationManager eine Zustandsverwaltung für die HCE Funktionalität. 

Nachdem auf einem Android-Gerät mehrere Applikationen vorhanden sein können, die eine HostApduService Komponente enthalten, muss das System eine Möglichkeit besitzen, die korrekte Instanz für ein ankommendes Command-Apdu auszuwählen. Um dies zu realisieren, besitzt der HostEmulationManager ein RegisteredAidCache-Objekt. Im Prinzip ist diese Komponente ein Speicherort, der registrierten AIDs die zugehörigen HostApduService-Instanzen zuordnet und diese verwaltet. Wird ein SELECT APDU registriert, wird daraufhin die passende HostApduService Instanz aus dem RegisteredAidCache geladen und alle nachfolgenden Apdus werden an dieses Service weitergeleitet.

Die Methoden des HostEmulationManagers zum Aktivieren der HCE Funktionalität sowie zur Weiterleitung von Command-Apdus werden von der sich im selben Paket befindlichen CardEmulationManager-Klasse aufgerufen. An diese Komponente werden ebenfalls die Response-Apdus zurückgeleitet. Der CardEmulationManager stellt eine zentrale Schnittstelle für unterschiedliche Services, die Card Emulation implementieren, zur Verfügung. In dieser Klasse findet die tatsächliche Verwaltung der HCE Services sowie der AIDs statt, die im HostEmulationManager ausgelesen werden. 

Der nächste Schritt in der Implementierung führt von der CardEmulationManager-Komponente zur NfcService Klasse. Aus dieser äußerst umfangreichen Klasse ist zu erkennen, dass diese die zentrale Instanz der NFC Applikation darstellt. Sie ist verantwortlich für die Ausführung jeglicher NFC Funktionalität auf dem Gerät und zur Weiterleitung von NFC Signalen an die zur Verarbeitung verantwortlichen Komponenten. Das NfcService implementiert das DeviceHostListener Interface, das über das Java Native Interface (JNI) von der nativen Implementierung der NFC Schnittstelle angesprochen wird. 

Über die native Implementierung werden die NFC Signale schlussendlich über die Hardware Abstraction Layer (HAL) Schicht und den Linux Kernel an die NFC Hardware weitergeleitet. Die Implementierung der nativen sowie der darunterliegenden Schichten liegt allerdings außerhalb des Rahmens dieser Diskussion und wird nicht weiter beleuchtet.
Die Implementierung der Umleitung der NFC Signale erfolgt ohnehin in den Java-Komponenten der NFC Schnittstelle und wird im nächsten Kapitel erläutert. 

\subsection{Implementierung der Interception}

In diesem Kapitel wird die eigentliche Problemlösung in einem oder mehreren Unterkapiteln ausgeführt. Die Strukturierung dieser Kapitel ist naturgemäß sehr stark von der konkre Aufgabenstellung abhängig. Der Name dieses Kapitels ist anzupassen, z.B. Umfeldbeschreibung -- Fallbeispiel \dots, konkreter schreiben je nach Art Diplomarbeit/Fragestellung.
\makeatletter\ifthesis@masterthesis
Nachfolgend einige Beispiele für unterschiedliche Arten von Diplomarbeiten.

Bei einer Software-Entwicklungsarbeit bieten sich folgende Unterkapitel an:
\begin{itemize}
	\item Im Kapitel \enquote{Design} sollte die konzeptionelle Lösung vorgestellt, diskutiert und begründet werden. Das Ergebnis dieses Kapitels könnte beispielsweise eine Protokoll-Architektur sein.
	\item Im Kapitel \enquote{Modelle} erfolgt üblicherweise das Feindesign. In diesem Kapitel könnten beispielsweise einzelne Protokolle bzw. Algorithmen aus der vorher definierten Protokoll-Architektur eingeführt und diskutiert werden. Achtung: Generell darauf achten, bei der eingangs erläuterten Notation zu bleiben und nicht Synonyme zu verwenden, verwirrt den Leser.
	\item Das Kapitel \enquote{Implementierung} sollte sich dann vorwiegend mit den Details der Umsetzung befassen. In diesem Kapitel sollte nur im Ausnahmefall exemplarisch Quellcode vorgesehen werden. Vielmehr sollten alle Probleme, die bei der Realisierung aufgetreten sind, dokumentiert, interpretiert und die Lösung erläutert werden.
\end{itemize}

Bei einer Arbeit zu einem abstrakteren Thema, bei dem ein oder mehrere Fallbeispiele aus der industriellen Praxis bearbeitet werden, bieten sich folgende Unterkapitel an:
\begin{itemize}
	\item Im Unterkapitel \enquote{Analyse der Problemstellung} wird die konkrete Problemstellung (die Situation im betrachteten Unternehmen) der Fallbeispiele beschrieben. Das Ergebnis dieses Kapitels könnte eine schematische Netzwerk- oder Applikationsarchitektur sein.
	\item Im Unterkapitel \enquote{Fallbeispiel} sollte sich (analog zur Implementierung in der Software-Entwicklung) mit den konkreten Details der Umsetzung befassen. Hier wird dargelegt, wie das zuvor identifizierte Lösungsschema konkret zur Anwendung gelangen kann bzw. welche Probleme während des Umsetzungsprojekts aufgetreten sind.
\end{itemize}

Bei einer Arbeit, deren Grundlage eine Auswahl eines Softwaresystems ist, bieten sich folgende Unterkapitel an:
\begin{itemize}
	\item IST-Analyse
	\item Hardware und Softwareausstattung
	\item Beschreibung der Geschäftsprozesse
	\item Schwachstellenanalyse des Unternehmens
	\item SOLL-Konzeption
	\item Auswahlverfahren möglicher verfügbarer Systeme -- Kriterienkatalog
	\item Einführung des neuen Systems
\end{itemize}

Bei einer Arbeit, deren Fokus auf der Durchführung und Auswertung von Fragebögen liegt, bieten sich folgende Unterkapitel an:
\begin{itemize}
	\item Im Kapitel \enquote{Problemstellung und Fragebogendesign} wird die fachliche Problemstellung detailliert erläutert und der Inhalt des Fragebogens in Bezug zur Problemstellung dargestellt.
	\item Im Kapitel \enquote{Befragungsmethode} werden die Untersuchungsobjekte (z.B. Praktische Ärzte), die Grundgesamtheit (Anzahl praktische Ärzte in Venezuela), Stichprobengesamtheit und das Verfahren zur Stichprobenziehung und das Erhebungsverfahren (Verteilung und Rücklauf der Fragebögen) beschrieben.
	\item Im Kapitel \enquote{Auswertungsmethode} werden die möglichen Auswertungsmethoden aufgelistet und ggf. begründet die ausgewählte Methode beschrieben.
	\item Im Kapitel \enquote{Befragungsdurchführung} wird die Untersuchungsdurchführung (z.B. Zeit, Ort der Befragung, Zeitraum der gesamten Befragung, besondere für das Untersuchungsergebnis oder zukünftige Forschungsarbeiten relevante Vorkommnisse etc.) dargestellt.
\end{itemize}

Hier intensive Rücksprache mit Ihren jeweiligen Fachbetreuern halten, mehrere Diplomarbeiten der Fakultät zu diesem Themenbereich durchsehen. Unabhängig vom Typ der Diplomarbeit werden im nachfolgenden Kapitel die konkreten Ergebnisse beschrieben.
\fi\makeatother