\makeatletter\ifthesis@masterthesis
%%%%%%%%%%%%%%%%%%%%%%%%%%%%%%%%%%%%%%%%%%%%%%%%%%%%%%%%%%%%%%%%%%%%%%%%
\chapter{Implementierung der Relay Attacke}
\label{sec:implementation}
%%%%%%%%%%%%%%%%%%%%%%%%%%%%%%%%%%%%%%%%%%%%%%%%%%%%%%%%%%%%%%%%%%%%%%%%

\section{ der Implementierung}

Zur Durchführung der NFC Relay Attacke wurden zwei handelsübliche, NFC-fähige Android Smartphones für die Rolle des Angreifers sowie des Opfers verwendet. Auf dem Angreifer-Gerät, einem HTC U11, war hierbei die Android Version 8.0.0 (API 26) vorhanden, auf dem Opfer-Gerät, einem Google Nexus 5X wurde eine modifizierte Version des Android Open Source Projektes mit der Version \textbf{XXX} installiert. Welche Modifikationen an diesem Betriebssystem vorgenommen wurden, wird in Kapitel \textbf{XXX} detailliert beschrieben.

Umgesetzt wurde die Relay Attacke durch die Entwicklung einer Proof-of-Concept Hostcard Emulation App, die eine handelsübliche mobile Payment Applikation wie beispielsweise Android Pay simulieren soll. Diese Anwendung wurde auf dem Opfer-Gerät installiert und als Ziel der Attacke sollte die HCE App zur Durchführung einer Zahlung und Weiterleitung der Daten an den Angreifer angeregt werden, ohne dass sich das Opfer in der Nähe eines POS Terminals befindet. Die Entwicklung der HCE App wird in Kapitel \textbf{XXX} dargelegt. 

Für die 

Die Kommunikation der Geräte wurde durch eine weitere mobile An
\fi\makeatother